\chapter*{What makes for good Gherkin?}

Read these two scenarios and underline terms that might be part of the team's ubiquitous language (remember that the ubiquitous language should be based on the business domain):

\begin{verbatim}
    Scenario: Dispense cash in multiple denominations
        Given I have $100 in my account
        And I have a card with the PIN 5173
        And I push my card into the machine
        And I enter 5173 for my PIN
        And I push the button next to “Withdrawal”
        And I push the button next to “Checking”
        When I push the button next to “$50”
        Then a $20 bill should be ejected by the cash dispenser
        And a $20 bill should be ejected by the cash dispenser
        And a $10 bill should be ejected by the cash dispenser
      
    Scenario: Dispense cash in multiple denominations
        When $50 is dispensed by the ATM
        Then the following bills should be received:
          | count | denomination |
          | 2     | $20          |
          | 1     | $10          |
    
\end{verbatim}

Discuss with your table group:

Both these scenarios describe the same behaviour. Which one do you think best expresses the essence of this behaviour from the business perspective?

\answerbox{1.5}

What's good about each of them? What's bad about each of them?

\answerbox{1.5}

If we were asked to change the system's UI, what effect would that have on these scenarios?
How might people in different roles see this differently? Why?

\answerbox{1.5}