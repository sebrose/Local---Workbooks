\chapter*{Start with the end in mind}

When we practice Behaviour-Driven Development (BDD), we take time before we start development to describe how we want the system to behave when we're done. We call this \textit{starting with the end in mind}.

Think about your own requirements for this workshop. What outcome would you like to see in your workplace as a result of this training? What would success look like from your perspective?

Write down some of these thoughts on sticky notes. Try to stick to one thought per note. Try to keep these ideas as concrete as possible.

Examples:
\begin{itemize}
    \item \textit{be able to explain the benefits of BDD to one of my colleagues}
    \item \textit{learn whether my team are doing BDD right or not}
    \item \textit{identify changes I can make to enable my teams to adopt continuous delivery}
    \item \textit{know where I can go to learn more about this subject}
\end{itemize}

