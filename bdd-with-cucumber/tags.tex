\chapter*{Tags}

\ifnotes

    Learning outcomes:

    \begin{enumerate}
        \item Add a \texttt{@wip} tag to one of the scenarios
\CYBERDOJO{
        \item Modify the \texttt{java} invocation in \texttt{cyber-dojo.sh} so that the start of the line looks like: \\
        \texttt{java  -Dcucumber.options="-{}-tags @wip" -cp \$CLASSES }
        \item Press the \emph{TEST} button
}{
        \item Open a command line for you project
        \item Type \texttt{mvn test -Dcucumber.options="-{}-tags @wip"} at the command line, and press enter
}
        \item Cucumber will now only execute the scenario you tagged as \texttt{@wip}
    \end{enumerate}


\fi

\ifcontent
    \CUKE{} allows you to add free-text \emph{tags} to your features, scenarios, scenario outlines and example tables (see \emph{Scenario Outline} sections):

    \begin{itemize}
        \item tags are an '@' character followed by some text (without any white space)
        \item tags applied to a feature apply to every scenario in that feature
        \item tags are case sensitive
        \item you can apply as many tags to a feature/scenario as you wish
    \end{itemize}

    \JAVA{
        You can then tell \CUKE{} which scenarios you want to run by supplying a \emph{tag expression} at runtime - either on the command line or via the \texttt{@CucumberOptions} Java annotation. Tag expressions are boolean and offer the logical operations \emph{and}, \emph{or} and \emph{not}.

        Let's use a tag to just run a single "Work in Progress" scenario.

        \begin{enumerate}
            \item Add a \texttt{@wip} tag to one of the scenarios
            \CYBERDOJO{
                    \item Modify the \texttt{@CucumberOptions} annotation in \texttt{RunCukesTest.java} so that it begins like: \\
                    \texttt{@CucumberOptions(tags=\{"@wip"\}, strict=true, ... )}
            }{
                    \item Open a command line for you project
                    \item Type \texttt{cucumber -{}-tags @wip} at the command line, and press enter
            }
            \item Cucumber will now only execute the scenario you tagged as \texttt{@wip}
        \end{enumerate}
    }

    \RUBY{
        You can then tell \CUKE{} which scenarios you want to run by supplying a \emph{tag expression} at runtime - either on the command line or via \emph{@CucumberOptions}. Tag expressions are boolean and offer the logical operations \emph{and}, \emph{or} and \emph{not}.

        Let's use a tag to just run a single "Work in Progress" scenario.

        \begin{enumerate}
            \item Add a \texttt{@wip} tag to one of the scenarios
            \item Open a command line for you project
            \item Type \texttt{cucumber -{}-tags @wip} at the command line, and press enter
            \item Cucumber will now only execute the scenario you tagged as \texttt{@wip}
        \end{enumerate}
    }

    \JAVASCRIPT{
        You can then tell \CUKE{} which scenarios you want to run by supplying a \emph{tag expression} at runtime. Tag expressions are boolean expressions and offer the logical operations \emph{and}, \emph{or} and \emph{not}.

        Let's use a tag to just run a single “Work in Progress” scenario.

        \begin{enumerate}
            \item Add a \texttt{@wip} tag to one of the scenarios
        \CYBERDOJO{
                \item Add the following line to the \texttt{cucumber} invocation in \texttt{cyber-dojo.sh} :\\
                \texttt{-{}-tags @wip \textbackslash }
                \item Press the \emph{TEST} button
        }
        {        \item Open a command line for you project
                \item Type \texttt{cucumber -{}-tags @wip} at the command line, and press enter. \\
                Note: You can also run \texttt{npm test -{}- -{}-tags @wip}
        }
            \item \CUKE{} will now only execute the scenario you tagged as \texttt{@wip}
        \end{enumerate}

        \section*{Logical tag expressions}
        In some cases, you may want to run scenarios that have different tag names such as:

        \hspace*{1cm}\texttt{@ShoutHeard} \\
        \hspace*{1cm}\texttt{Scenario: In range shout is heard} \\
        \hspace*{1cm}\texttt{...}

        \hspace*{1cm}\texttt{@ShoutNotHeard} \\
        \hspace*{1cm}\texttt{Scenario: Out of range shout is not heard} \\
        \hspace*{1cm}\texttt{...}

        \CYBERDOJO{
            Modify \texttt{cyber-dojo.sh} so that the invocation of \CUKE{} starts off like this:

            \hspace*{1cm}\texttt{./cucumber.js} \\
            \hspace*{2cm}\texttt{-{}-tags "@ShoutHeard or @ShoutNotHeard" \textbackslash}
            }
            {
            \hspace*{1cm}\texttt{npm test -{}- -{}-tags "@ShoutHeard or @ShoutNotHeard"}
            }
    }

    \CSHARP{
        \CUKE{} converts the tags into "Traits" or "Categories" which you can use to selectively run scenarios in your test-runner.

        \CYBERDOJO{}{Lets use the VisualStudio Test Explorer window to just run a single "Work in Progress" scenario.}

        \begin{enumerate}
            \item Add a \texttt{@wip} tag to one of the scenarios
            \item Run \CUKE{} to confirm all scenarios are still green
        \CYBERDOJO{
                \item Modify the \texttt{mono} invocation at the bottom of \texttt{cyber-dojo.sh} so that the start of the line looks like: \\
                \texttt{mono \${NUNIT_RUNNERS_PATH}/nunit3-console.exe  --where "cat==wip" --noheader }
                \item Press the \emph{TEST} button
        }
        {
                \item Open the Test Explorer window
                \item Type \texttt{Trait:wip} in to the Search text box, and press enter
                \item 'Run all' will now only execute the scenario you tagged as \texttt{@wip}
        }
        \end{enumerate}
    }
\fi
