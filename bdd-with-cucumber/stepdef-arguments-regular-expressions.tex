\chapter*{Step definition arguments}

\ifnotes

    Learning outcomes:
    
    \begin{itemize}
        \item Explain simple regex strings
        \item Utilise regex to reduce the number of stepdefs
        \item Understand the what happens if more than one regex matches a step
    \end{itemize}
    
    We want people to replace \texttt{Lucy is at} with \texttt{(Lucy|Sean) is at} and see the error caused by the ambiguity of having more than one regex match a step.
    
    Regex crib sheet:
    
    \begin{itemize}
        \item \CARET - match the start of a string
        \item \DOLLAR - match the end of a string
        \item Character classes:
            \begin{itemize}
                \item \BACKSLASH d - digit
                \item \BACKSLASH w - word character (a-z A-Z 0-9 \textunderscore)
                \item \BACKSLASH s - whitespace
            \end{itemize}
        \item ( ) - capture group. Each capture group will be passed as an argument to the step definition
        \item . - wild card, matches any character
        \item Multiplicity operators:
            \begin{itemize}
                \item * - 0 or more
                \item + - 1 or more
                \item ? - 0 or 1
            \end{itemize}
    \end{itemize}

    Things you might want to point out:
     \begin{itemize}
        \item mismatch between number of capture groups and parameters on stepdef will cause an error
        \item the order of the capture groups must match the order of the parameters
        \item capture groups capture STRINGS, but (in typed languages) we might be receiving \texttt{int} - Cucumber is doing type coercion
        \item if more than one regex matches the step, then Cucumber treats this ambiguity as an error
        \item there's a trade off between using an \emph{alternate} and a \emph{character class}. Alternates may lead to more maintenance, but can help restrict documentation to known persona. Character class are more forgiving, but mistyping can cause problems
        \item regex is a strict match - an extra or missing space can cause the match to fail unexpectedly, which can cause confusion
     \end{itemize}
     
\fi 

\ifcontent 
    We're already passing integer arguments into our \JAVA{\texttt{sean_is_at} }\CSHARP{\texttt{GivenSeanIsAt}}\JAVASCRIPT{\texttt{"Sean is at"} }\RUBY{\texttt{"Sean is at"} } and \JAVA{\texttt{lucy_is_at} }\CSHARP{\texttt{GivenLucyIsAt} }\JAVASCRIPT{\texttt{"Lucy is at"} }\RUBY{\texttt{"Lucy is at"} } step definitions using regular expression \emph{capture groups}. These steps are almost identical except for the person's name. Let's replace the person's name in the regular expressions with another \emph{capture group} so that the person's name is also passed to the step definition as a parameter.
    
    Take the regular expressions of one of the existing step definitions, and wrap the person's name in parentheses to make it a capture group:
    
    \JAVA{\texttt{@Given("\CARET{}(Lucy) is at (\DOUBLEBACKSLASH{}d+), (\DOUBLEBACKSLASH{}d+)\DOLLAR{}")}}
    \CSHARP{\texttt{[Given(@"(Lucy) is at (.*), (.*)")]}}
    \JAVASCRIPT{\texttt{Given(/\CARET{}(Lucy) is at (\BACKSLASH{}d+), (\BACKSLASH{}d+)\DOLLAR{}/, function (x, y, callback) \LEFTCURLY{}}}
    \RUBY{\texttt{Given(/\CARET{}(Lucy) is at (\BACKSLASH{}d+), (\BACKSLASH{}d+)\DOLLAR{}/ do |x, y|}}
    
    Now that the regular expression has 3 \emph{capture groups} the step definition method is going to need 3 parameters. What type should the new parameter be? What's a good name for it?
    
    Can you see where you can now replace the hard-coded "Lucy" in the body of the step definition method with that parameter?
    
    Run the tests and check everything is still working as before.
    
    Now add Sean as an \emph{alternate} in the regular expression capture group for the person's name:
    
    \JAVA{\texttt{@Given("\CARET(Lucy|Sean) is at (\DOUBLEBACKSLASH d+), (\DOUBLEBACKSLASH d+)\DOLLAR")}}
    \CSHARP{\texttt{[Given(@"(Lucy|Sean) is at (.*), (.*)")]}}
    \JAVASCRIPT{\texttt{Given(/\CARET(Lucy|Sean) is at (\BACKSLASH d+), (\BACKSLASH d+)\DOLLAR/, function (person, x, y, callback)}}
    \RUBY{\texttt{Given(/\CARET{}(Lucy|Sean) is at (\BACKSLASH{}d+), (\BACKSLASH{}d+)\DOLLAR{}/ do |person, x, y|}}
    
    Run \CUKE{}. Did you get \JAVA{an \texttt{AmbiguousStepDefinitionsException}? }\JAVASCRIPT{a \texttt{Multiple step definitions match}\ error message? }\RUBY{an \texttt{Ambiguous match}\ error message? }\CSHARP{a \texttt{BindingException } because of an ambiguous step definition? } Why? What do you need to do to fix it?
    
    \answerbox{1}
    
    \section*{Step definition matching}
    
    \CUKE{} uses standard regular expressions to match steps to step definitions. Take a look at the regular expressions in the step definition file. Try to explain what these regular expression characters mean:
    
    \begin{itemize}
        \JAVA{\item \texttt{\CARET}}
        \RUBY{\item \texttt{\CARET}}
        \JAVASCRIPT{\item \texttt{\CARET}}
        \item \texttt{\DOLLAR}
        \item \texttt{\JAVA{\DOUBLEBACKSLASH} \CSHARP{\BACKSLASH} \JAVASCRIPT{\BACKSLASH}\RUBY{\BACKSLASH}d}
        \item \texttt{(   )}
        \item \texttt{.}
        \item \texttt{\PLUS}
        \item \texttt{*}
        \item \texttt{Lucy|Sean}
    \end{itemize}
\fi    
