\chapter*{\CUKE{} API}

\ifnotes

    Learning outcomes:
    
    \begin{itemize}
        \item  
    \end{itemize}

\fi 

\ifcontent 
    Gherkin has 5 keywords that we use to introduce steps: \texttt{Given}, \texttt{When}, \texttt{Then}, \texttt{And}, \texttt{But}.
    
    If you look in our feature file you'll see that we use 4 of them already. Now look in the step definition file - how many different \JAVASCRIPT{methods}\JAVA{annotations}\CSHARP{attributes}\RUBY{methods}\ can you find?
    
    \answerbox{1}
    
    Why is that?
    
    \answerbox{2.5}
    
    
    \JAVA{
        Replace all the \texttt{@Given} and \texttt{@When} annotations in the step definition with \texttt{@Then}. Run Cucumber again. All the steps still pass. Why?
        
        \answerbox{2.5}
        
        Now go to the feature file and replace every \texttt{Given}, \texttt{When} and \texttt{Then} with \texttt{And}. Run \CUKE{} again. All the steps still pass. Why?
        
        \answerbox{2.5}
    
        Remember, \CUKE{} is a collaboration tool and its purpose is for the team to write acceptance tests in the ubiquitous language so that everyone can understand them. The keywords (\texttt{Given}, \texttt{When}, \texttt{Then}, \texttt{And}, \texttt{But}) exist so that you can write easy to understand scenarios, but as far as \CUKE{} is concerned they all mean the same thing.
    
        Now undo all those edits, so that the scenarios make sense to us again.
    }
    
    \JAVASCRIPT{
        Replace all the \texttt{Given()} and \texttt{When()} method calls in the \texttt{defineSupportCode} call with \texttt{Then()}. Run Cucumber again. All the steps still pass. Why?
        
        \answerbox{2.5}
        
        Now go to the feature file and replace every \texttt{Given}, \texttt{When} and \texttt{Then} with \texttt{And}. Run \CUKE{} again. All the steps still pass. Why?
        
        \answerbox{2.5}
    
        Remember, \CUKE{} is a collaboration tool and its purpose is for the team to write acceptance tests in the ubiquitous language so that everyone can understand them. The keywords (\texttt{Given}, \texttt{When}, \texttt{Then}, \texttt{And}, \texttt{But}) exist so that you can write easy to understand scenarios, but as far as \CUKE{} is concerned they all mean the same thing.
    
        Now undo all those edits, so that the scenarios make sense to us again.
    }
    
    \RUBY{
        Replace all the \texttt{Given()} and \texttt{When()} method calls with \texttt{Then()}. Run Cucumber again. All the steps still pass. Why?
        
        \answerbox{2.5}
        
        Now go to the feature file and replace every \texttt{Given}, \texttt{When} and \texttt{Then} with \texttt{And}. Run \CUKE{} again. All the steps still pass. Why?
        
        \answerbox{2.5}
    
        Remember, \CUKE{} is a collaboration tool and its purpose is for the team to write acceptance tests in the ubiquitous language so that everyone can understand them. The keywords (\texttt{Given}, \texttt{When}, \texttt{Then}, \texttt{And}, \texttt{But}) exist so that you can write easy to understand scenarios, but as far as \CUKE{} is concerned they all mean the same thing.
    
        Now undo all those edits, so that the scenarios make sense to us again.
    }
    
    \CSHARP {
        Change \texttt{And Sean is at 0, 900} in the feature file to \texttt{When Sean is at 0, 900} and run \CUKE{}. What happens now?
    
        \answerbox{2.5}
    
        How does \CUKE{} decide which step definition to run for \texttt{And} steps?
    
        \answerbox{2.5}
    
        Now undo that edit, so that the scenarios make sense to us again.
    }
\fi    
