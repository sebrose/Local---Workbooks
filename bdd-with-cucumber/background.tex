\chapter*{Backgrounds}

\ifnotes

    Learning outcomes:
    
    \begin{itemize}
        \item Describe how a background works within the lifecycle of a scenario
        \item Explain that backgrounds damage locality of reference
    \end{itemize}

    The steps in a background will be run before running each scenario in the current feature file. 
    
    A background forces the reader of the documentation to remember that it's there. This can be problematic. Avoid them.
\fi 

\ifcontent 
    Sometimes each scenario in a feature file repeats exactly the same \texttt{Given} step(s). It can be useful to remove this duplication by placing them in a \texttt{Background} which gets run before \underline{each} scenario. In our example Lucy is always at \texttt{ 0, 0 } so we could move this step into a background.
    
    \begin{verbatim}
      Background:
        Given Lucy is at 0, 0
    \end{verbatim}
    
    Create this background in your feature file and remove the references to "\texttt{Lucy is at 0, 0}" from all the scenarios. Run \CUKE{} again to make sure it's all still working.
    
    Remember, though, that the purpose of the feature file is to make the specification of the system easy to read. Moving steps out of a scenario and into a background can make it harder to read the feature file, because now each scenario now needs to be read in conjunction with the background.
    
    \TRAINERNOTES{
        Is the feature file really easier to read now?
        
        I discourage students from using backgrounds, because it harms 'locality of reference'. When reading the feature file, you always have to remember what's at the top.
    }
\fi