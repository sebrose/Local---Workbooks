We need some way to share state between our step definition classes, while at the same time maintaining isolation between our scenarios. \CUKE{} encourages you to do this by using a Dependency Injection (DI) container. \CUKE{} doesn't have its own DI container, but it ships with linkages to several common containers, including Spring.

The problem you encountered when splitting the existing step definitions between 2 classes was sharing the Shouty instance. We can use Spring to solve this problem:

%     <bean class="Shouty" autowire-candidate="true" scope="cucumber-glue" />


\begin{itemize}
    \CYBERDOJO
        {}
        {\item Annotate the \texttt{Shouty} class with \texttt{@Component}}
    \item For the \texttt{Shouty} instance field in \underline{both} step definition classes 
    \begin{enumerate}
        \item Annotate with \texttt{@Autowired}
        \item Modify the line following the annotation to read: \texttt{private Shouty shouty;}
    \end{enumerate}
\end{itemize}

These annotations give the responsibility for managing the \texttt{Shouty} instance to Spring. By default Spring will create a single instance of \texttt{Shouty} and inject it where-ever requested.

Run \CUKE{}. What happens? Why?

\answerbox{1}


\section*{Isolation}

Before we started using Spring we created the \texttt{Shouty} instance explicitly in the step definition class. As described in the \CUKE{} Lifecycle section, \CUKE{} deletes all step definition classes before each scenario runs. This meant that we got a fresh \texttt{Shouty} object for each scenario.

Now we have delegated management of the \texttt{Shouty} object to Spring, and it doesn't know about \CUKE{} and scenarios. Spring creates a singleton \texttt{Shouty} object and reuses it for each scenario, which is allowing state to leak. We need to tell Spring to create a new \texttt{Shouty} object for each Scenario:

\CYBERDOJO
{
    \begin{itemize}
        \item Change the \texttt{bean} element in \texttt{cucumber.xml} to read:
        
        \texttt{<bean class="Shouty" autowire-candidate="true" scope="cucumber-glue" />}
    \end{itemize}

    We have modified the Spring configuration to associate autowired \texttt{Shouty} objects with the \\ \texttt{cucumber-glue} scope. The Cucumber-Spring integration defines the \texttt{cucumber-glue} scope and uses it to reset state before each scenario.
}
{
    \begin{itemize}
        \item Annotate the \texttt{Shouty} class with \texttt{@Scope("cucumber-glue")}
    \end{itemize}

    This is a Spring annotation, but the Cucumber-Spring integration defines the \texttt{cucumber-glue} scope and uses it to reset state before each scenario.
}


Once you have modified the code, run \CUKE{} again. All scenarios should now pass.

% REMOVE THIS CHAPTER

\chapter*{Another level of indirection}

You may have noticed that we added annotations to the \texttt{Shouty} object, which is our production code. Obviously we can't modify production code just to help our test code, so we'll need another solution.

There's a famous computer science quote by David Wheeler, which says:

\setlength{\leftskip}{1cm}

    \textit{"All problems in computer science can be solved by \textbf{another level of indirection}"}

\setlength{\leftskip}{0pt}

And that's just what we're going to do here. 
\begin{enumerate}
    \item Remove the \texttt{@Component} and \texttt{@Scope} annotations from \texttt{Shouty}
    \item Create a \texttt{ShoutyHelper} class in the step definition folder that \texttt{extends} \texttt{Shouty}
    \item Annotate \texttt{ShoutyHelper} with \texttt{@Component} and \texttt{@Scope("cucumber-glue")}
    \item Replace direct references to \texttt{Shouty} in the step definition classes with references to\\ \texttt{ShoutyHelper}
\end{enumerate}

Run \CUKE{} again. All scenarios should still be passing.