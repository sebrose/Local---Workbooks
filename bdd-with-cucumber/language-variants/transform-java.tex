We saw how \CUKE{} will try to convert the string matched in a regular expression \emph{capture group} into an object of the type specified by the step definition. Let’s look at this more closely.

\section*{Constructor conversion}

One of the mechanisms Cucumber will use is to call a public constructor that takes a single \texttt{String} argument. Let's try this out using the \texttt{Coordinate} class:

\begin{enumerate}
    \item Add a new constructor to \texttt{Coordinate}:

\begin{verbatim}
    public Coordinate(String coordinate){
        String[] component = coordinate.split(",");
        x = Integer.valueOf(component[0].trim());
        y = Integer.valueOf(component[1].trim());
    }
\end{verbatim}


    \item Change your step definition from:

\begin{verbatim}
    @Given("^(\\w+) is at (\\d+), (\\d+)$")
    public void person_is_at(String person, int x, int y) throws   
                                                      Throwable {
        shouty.setLocation(person, new Coordinate(x, y));
    }    
\end{verbatim}

To:

\begin{verbatim}
    @Given("^(\\w+) is at (.*)$")
    public void person_is_at(String person, Coordinate coordinate)   
                                               throws Throwable {
        shouty.setLocation(person, coordinate);
    }
\end{verbatim}
\end{enumerate}

Run Cucumber. All scenarios should still be green.



\chapter*{Using a \texttt{Transformer}}

In the previous example we modified the \texttt{Coordinate} class to give it a new constructor. This class was \emph{production} code and we're often not able to change production code. Instead we could use a \emph{transform}.

\CYBERDOJO
    {
        \textbf{HINT: the \texttt{Transform} annotation and the \texttt{Transformer} class both live in the \texttt{cucumber.api} package.}
    }
    {}

\begin{enumerate}
    \item Change the step definition:

\begin{verbatim}
    @Given("^(\\w+) is at (.*)$")
    public void person_is_at(String person,
    @Transform(CoordinateConverter.class) Coordinate coordinate) throws 
                                                             Throwable {
        shouty.setLocation(person, coordinate);
    }
\end{verbatim}

\item Create a new class, CoordinateConverter, in the step definition folder:

\begin{verbatim}
    public class CoordinateConverter extends Transformer<Coordinate> {
       public Coordinate transform(String input){
           String[] component = input.split(",");
           int x = Integer.valueOf(component[0].trim());
           int y = Integer.valueOf(component[1].trim());
    
           return new Coordinate(x, y);
       }
}
\end{verbatim}

\item Delete the String constructor that we added to Coordinate.
\end{enumerate}

Run \CUKE{}. All scenarios should still be green.

