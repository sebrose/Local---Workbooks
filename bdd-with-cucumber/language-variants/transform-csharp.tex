We saw how \CUKE{} will try to convert the string matched in a regular expression \emph{capture group} into an object of the type specified by the step definition. Let’s look at this more closely.

\section*{StepArgumentTransformation}

\CUKE{} is able to automatically convert captured parameters to built-in types, like \texttt{int} or \\ \texttt{DateTime}, but what about our \texttt{Coordinate} class?

\begin{itemize}
    \item Change the step definition:

        \begin{verbatim}
            
        
        [Given(@"(.*) is at (.*)")]
        public void GivenLucyIsAt(string name, Coordinate coordinate)
        {
            shouty.SetLocation(name, coordinate);
        }
        \end{verbatim}

\item Create a new method in the \texttt{LocationSteps} class, \texttt{ConvertCoordinate}, in the step definition folder, and annotate it with the \texttt{[StepArgumentTransformation]} attribute below:

    \begin{verbatim}
        [StepArgumentTransformation(@"(.*), (.*)")]
        public Coordinate ConvertCoordinate(int xCoord, int yCoord)
        {
            return new Coordinate(xCoord, yCoord);
        }
        
    \end{verbatim}
\end{itemize}    

Run \CUKE{}. All scenarios should still be green.


You can now add different \texttt{StepArgumentTransformations} which will deal with place names rather than coordinates, e.g. North Pole is at \texttt{0, 0}.

See if you can add the transformation and change Lucy's location to the North Pole.
