\ifnotes

    \chapter*{Introductory training notes}

    Do regular coding on the screen to help the room get unstuck:
    
    \begin{itemize}
        \item Do it mob-programming style
        \item You type, and ask attendees to tell you what to type
    \end{itemize}

    Half of attendees are very inexperienced programmers. You cannot expect them to complete exercises with just verbal explanation. They need to see the solution, or else they will be lost from very early on.
    
    \begin{itemize}
        \item 
            \item Set up a starter project, just like attendees, at the beginning of the day
            \item For every exercise, when half the people are done, ring the bell
            \item Show your project, which hasn’t solved the exercise
            \item Ask the room how to solve the exercise. Ask them to tell you what to type.
            \item While you’re typing, ask them to explain what is going on.
            \item When done, leave it on the screen so laggards can copy.
            \item Tell the room to continue and spend time with laggards to make sure they catch up
    \end{itemize}

    By the end of the day you will have gradually evolved the codebase and everyone will have seen how it evolved.
    
    Every evolution on the projector is a great way to discuss what, why and how with the room.
\fi 
