\chapter*{Introducing a seam}

We would rather not suffer from intermittent failures while test-driving our code, so we're going to try and replace the unreliable \texttt{\ShoutyStatsService} with something a bit more predictable. To do that, we're going to introduce a \emph{seam}.

\section*{What is a seam?}

Michael Feathers first used the word 'seam' in his book \textit{Working Effectively With Legacy Code}.

His definition is:

\begin{framed}
"A seam is a place where you can alter behavior in your program without editing in that place."
\end{framed}

The benefit of a seam is that, if you're working with a difficult or unreliable software component, you can replace it (during testing) with a test double that behaves the same way.

\section*{Wrapping \texttt{\ShoutyStatsService}}

We're going to put a seam between our code and the unreliable \texttt{\ShoutyStatsService}. The first step is to create an interface that defines the services that our report job wants to make use of. When we define this interface, it's important to think about how we'd ideally like it to be, rather than being influenced by the details of what it currently does. We call this \emph{wrapping} the service.

Ideally, we'd just get back a \texttt{\CSHARP{decimal}\JAVA{double}\CPP{double}} from the API, rather than a string of XML, so we'll express that in the interface. Take a look at the \texttt{\RevenueProviderInterface} \NOTCPP{interface}\CPP{abstract base class}.

Now we can create a concrete adapter that implements this interface, including the behaviour to extract the revenue value as a \texttt{\CSHARP{decimal}\JAVA{double}\CPP{double}} from the XML. Take a look at the \texttt{\ProductionRevenueProvider} class. You'll see that there's some code commented out in the class - uncomment it and delete the line that throws an exception.

So far, we've looked at some new code, but we haven't changed the existing \texttt{\ShoutyStatsService}, so if you run the tests they should all still pass (at least some of the time).

\section*{Using our interface}

\begin{sloppypar}
Now it's time to modify \texttt{\ShoutyReportProcessor} to use \texttt{\ProductionRevenueProvider} instead of \texttt{\ShoutyStatsService}. Remember that, although you'll need to create an instance of \texttt{\ProductionRevenueProvider} in \texttt{\ShoutyReportProcessor}, you should store and pass around the reference as a \texttt{\RevenueProviderInterface}.

There's some code to change, and you'll be able to delete the code that parses the XML returned by \texttt{\ShoutyStatsService} (because that's already implemented in  \texttt{\ProductionRevenueProvider}).

Run the tests. They should all pass (even if they are still flaky).
\end{sloppypar}

\section*{Create the seam}

What we've done is introduce \emph{another level of indirection}. \texttt{\ShoutyReportProcessor} now uses \texttt{\RevenueProviderInterface} to indirectly call \texttt{\ShoutyStatsService}, but we haven't solved our original problem. The report is still generated by using the unreliable, external service.

To change this we're going to use a technique called \emph{dependency injection}.

\begin{sloppypar}

Change the code in \texttt{\ShoutyReportJob} so that the \texttt{\RevenueProviderInterface} reference is created in \texttt{\ShoutyReportJob} and pass it to the \texttt{\ShoutyReportProcessor} constructor.  

Now change the \texttt{\ShoutyReportProcessor} constructor, so that it also takes a reference to \texttt{\RevenueProviderInterface}. This is where we will inject the dependency.
\end{sloppypar}

\JAVA{\import{testable-architecture/language-variants/}{seam-injection-java.tex}}
\CSHARP{\import{testable-architecture/language-variants/}{seam-injection-csharp.tex}}
\CPP{\import{testable-architecture/language-variants/}{seam-injection-cpp.tex}}

Run the tests again. They should all still pass (eventually).  

\section*{What did we just do?}

\QandAbox{Draw a diagram of the code now, including the changes you've made to introduce this seam.}{8}
