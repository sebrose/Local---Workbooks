\begin{verbatim}
    proc.StartInfo.EnvironmentVariables.Add("VOLATILE_STATS_DATA", 
                                            "true");
\end{verbatim}