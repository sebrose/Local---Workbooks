\chapter*{Appendix 1: Shouty Product Concept}

\ifnotes

    Get the whole team to read this. It describes some rules/acceptance criteria. It also says what the initial priority should be (use case).
    
    Most delegates won't have real-life experience of coding GPS, so the range aspects are probably the most risky \& hence interesting.
    
    Features/rules that often come up include:
    
    \begin{itemize}
        \item listener moving in \& out of range
        \item lifetime of a shout
        \item shouter wanting to locate the shout at place of business, not their current location
        \item listener being able to re-shout
    \end{itemize}

\fi 


\ifcontent

    We are developing a new social media app - with some similarities to Twitter - called Shouty. Users of the app will be able to 'shout' - and will be 'heard' by other users who are within 1000m of the shouter. 
    
    Shouty will initially support the following use case: 
    \begin{itemize}
        \item Local Business Promotions e.g. "Half price coffee at Barney's Café until 12 today"
    \end{itemize}
    
    The target platform is GPS-enabled smartphones.
    
    Functional Requirements:
    \begin{itemize}
        \item Shouts should be text only - limited to 2000 characters
        \item The range of shouts should be 1000m
    \end{itemize}
    
    Initial story:
    \begin{verbatim}
        As a business owner
        I want Shouty users within range to receive my promotional shouts
        So that I can generate more business
    \end{verbatim}

\fi
