\chapter*{Discovery workshop}

\ifnotes

    Learning outcomes:
        \begin{itemize}
            \item aka 3 Amigos or Discovery Workshop
            \item the preparation needed: PO/BA should notify team of topic at least the day before; bring along story and ACs
            \item the length and frequency: 30 mins max, every day (if possible)
            \item the necessary participants: at least one representative from business, dev, test
            \item expected outputs \& outcomes: concrete examples; refined or new rules; refined or smaller stories
            \item process of creating the cards: everyone should participate; devs and test lead
            \item number of rule cards will inform on story size: more than three => probably "too big"
            \item finer granularity for prioritisation: PO identifies rules that are priority, and splits stories accordingly
            \item the indicators that a story is not yet ready: question cards; owner for each card needed
            \item what happens to the cards => photograph, attached to JIRA or similar 
            \item remote example mapping: many tools, including MindMup           
        \end{itemize}    
        
        Also in the workbook,  is the How To Split A User Story graphic. Let them know it's freely available online. Looking at an example map is only one way to split a story.    

    Setup:
    
        This is run as a Concept Clinic (see Jigsaw Activities in TFTBOTR, or www.netspeedleadership.com) There are four concepts:
        
        \begin{itemize}
            \item Preparation and scheduling for the workshop: who organises it; how long should it run; how often; what pre-session notifications; what pre-session work
            \item Running a workshop: who comes; what are their responsibilities; keeping the session flowing
            \item Actions taken at the workshop:  multiple examples for each rule; judge story size; not well understood; split stories; re-prioritisation; ownership if questions
            \item Output from the workshop: where is everything captured; formulate each example or not; how about remote participants
        \end{itemize}
        
        Write/print out separate starting sheets for each concept above.
        
        One way of running this is with a sheet for each table/group, so this is ideal when the number of tables/groups is a multiple of 4. Assign each table/group a concept. Give them some time to discuss the topic and research/guess answers. They need to write their ideas on the concept sheet. At the end of the time period, pass the sheet to the next table/group. Continue till each table/group has discussed each concept. 
        
        An alternative, that I prefer, is to create a single flip chart sheet for each concept. Divide each sheet into scetions - one section per question you want them to discuss. Get them to form into groups around each flip chart and discuss the topics, writing their answers onto the sheet. Then rotate. At the end, go round the four concepts in order reinforcing the learning.
        
        Then get them to do a short presentation on each..... and correct/reinforce as necessary. THIS IS WHEN THEY SHOULD START FILLING IN THE KEY FACTS/ANSWERS ON THE NEXT PAGE
\fi

\ifcontent
    A \emph{Discovery Workshop} is a time when people come together to collaborate and build their shared understanding of some part of the product that they are building. During this exercise think about meetings and workshops that you have attended. What made some of them more successful than others? 
    
    \textbf{Don't write anything on this page until AFTER the exercise.}\\
    
    \begin{framed}
        A. Preparing and scheduling an Discovery Workshop \\
           \\
           \\
           \\
           \\
           \\
           \\
           \\
           \\
        B. Running a Discovery Workshop \\
           \\
           \\
           \\
           \\
           \\
           \\
           \\
           \\
        C. Actions taken at a Discovery Workshop \\
           \\
           \\
           \\
           \\
           \\
           \\
           \\
           \\
        D. Output from a Discovery Workshop
           \\
           \\
           \\
           \\
           \\
           \\
           \\
           \\
    \end{framed}
\fi
