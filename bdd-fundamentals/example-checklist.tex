\chapter*{Good example checklist}

\ifnotes

    When attendees have read these, go through each item and ask the room: Why do you think this is important?

\fi 

\ifcontent

    \begin{enumerate}
        \item \textbf{Concrete} - use actual data values in your examples. If you find yourself using abstract descriptions (e.g. `amount requested') try replacing them with concrete values (e.g. \$40).
        \item \textbf{Essential} - any data that you do use in an example should contribute to understanding the behaviour being illustrated. Including inessential data makes it harder to read the example and understand what its purpose is.
        \item \textbf{Declarative} - examples that declare what the user is trying to achieve (e.g. `withdraw \$40') are easier to understand than those that describe imperatively what the user does (e.g. `enter \$40 through the keypad and click the "Withdraw" button')
        \item \textbf{Focused} - don't try to illustrate more than one interaction in any example. If you need more examples to fully explore the behaviour, then create them.
        \item \textbf{Ubiquitous} - avoid using terms that are only understood by some of the team. In general, try to use business terminology throughout.
        \item \textbf{Interesting} - every example should have a reason for existing. Try and give each one an intention revealing name.
    \end{enumerate}

\fi
