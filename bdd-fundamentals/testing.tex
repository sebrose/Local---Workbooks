\chapter*{Testing}

\ifnotes
    Learning outcomes:
    
    \begin{itemize}
        \item Describe how requirements can be tested using examples
        \item Explain why not all testing can be automated
        \item Understand that testing is a broader activity than typically understood
    \end{itemize}
        
    Delegates often don't find much to discuss on this poster. Direct them to the Jerry Weinberg quote and ask them to consider his definition of testing when discussing the questions.
    
    You may need to directly engage them in conversation for the last minute or two of their time at this concept centre, to help them realise that testing is broader than typically understood.
    
    When reviewing this slide, it's useful to refer to Elisabeth Hendrickson's idea that testing is a combination of Checking (that can be automated) and Exploring (which can't be automated)
\fi 

\ifcontent
    \begin{framed}
       "... the process of gathering information about something with the intent that the information could be used for some purpose …" 
       
       \begin{flushright}
            \textit{Jerry Weinberg, Perfect Software (and Other Illusions About Testing)}
        \end{flushright}
    \end{framed}    
    
    \vspace{0.5cm}
    \QandAbox{Is code the only thing that can be tested?}{2}
    
    \QandAbox{Can all testing be automated?}{2}
    
    \QandAbox{How do you test that the requirements are themselves correct?}{2}
    
    \QandAbox{What do you think of Jerry Weinberg's definition of 'Testing'?}{2}
\fi