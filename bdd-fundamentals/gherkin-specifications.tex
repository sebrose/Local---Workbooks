\chapter*{Gherkin Specifications}

\ifcontent

    Gherkin is the name of the specification language that Cucumber understands. It's designed to be just formal enough to be read by a computer, but flexible enough that people can use it too.
    
\fi 

\section*{Given / When / Then}

\ifnotes

    Alternative: get tables to do this exercise as a group

    Learning outcomes:
    
    \begin{itemize}
        \item Explain that Gherkin is an easy-to-read format
        \item Describe how terms in the ubiquitous language only need to be understood within your organisation (bounded context) - we're writing for a specific audience, not the world in general
    \end{itemize}
    
    
    Ask the room: Does anyone recognise this structure? (It's Context-Action-Outcome)

\fi

\ifcontent

    Here are some examples of Gherkin scenarios from real projects:

\begin{verbatim}
    Scenario: Withdrawal from account in credit reduces balance
      Given my account has a balance of $100
      When I withdraw $25
      Then my account balance should be $75
    
    Scenario: Create a new user account
      Given an unauthenticated session
      And I provide an acceptable username and password
      When I request a new account
      Then a new account should be created for me
\end{verbatim}

\fi

\chapter*{Practice Given / When / Then}

\ifnotes

    Learning outcomes:
    
    \begin{itemize}
        \item Explain the difference between Context/Action/Outcome
        \item Describe the use of conjunctions: And/But
        \item Explain why there is no conjunction: Or 
    \end{itemize}

\fi 


\ifcontent
    The following steps have been jumbled up. Write them out in the correct order.

\begin{verbatim}
    Given ...       I should have $75 in my account
    When ...        I have $100 in my account
    Then ...        I withdraw $25
      

    Given ...       I request to withdraw $25
    When ...        I have $10 in my account
    Then ...        I should still have $10 in my account
    

    Given ...       I request to withdraw $50
    And ...         I have an overdraft limit of $100
    When ...        I have $10 in my account
    Then ...        I should be overdrawn by $40 in my account


    Given ...       I request to withdraw $30
    And ...         The ATM does not hold any $10 bills
    When ...        I have $100 in my account
    Then ...        I should still have $100 in my account  
    
    
\end{verbatim}

    \QandAbox{Swap answers with someone at your table and check one another's answers. Did you give the same answers? If not, why not?}{5}

\fi