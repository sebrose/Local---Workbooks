\ifnotes

    \chapter*{Introductory training notes}

    These notes are intended to be useful for all Cucumber Limited training courses.
    
    \section*{Starting}
    
    Our courses usually start with a warm-up exercise, that get delegated "in the zone" while waiting for everyone to arrive.
    
    When it's time to start the course, there are a few things that you should tell the attendees:
    
    \begin{itemize}
        \item Practicalities: breaks, lunch, finish, toilets, fire
        \item Agree a method for getting the attention of the whole class - a bell, a hand signal
        \item Approach: experiential, very few slides => lots of exercises, so no snoozing
        \item For multi-day trainings, give them an idea about how the course is structured
        \item Explain why you are an ideal trainer - very briefly ;)
    \end{itemize}
    
    You will probably follow this introduction with a goal setting exercise.

    \section*{}
    
    \begin{itemize}
        \item Ask delegates to read from the workbook, to engage with the material
        \item Once they have completed an exercise ask them questions to check they've achieved the learning outcomes
        \item Only explain things once they have had a chance to answer
    \end{itemize}

    \section*{Ending}
    
    Ensure you have enough time at the end of the day to:
    
    \begin{itemize}
        \item Go through the warm-up questions
        \item Check that the delegates goals have been met (or not)
        \item Complete the Wrap Up exercise
    \end{itemize}

    
    \section*{Resources}
    
    \begin{itemize}
        \item Richard Lawrence's intro technique: http://agileforall.com/how-to-give-clear-instructions-when-facilitating/
    \end{itemize}

\fi 
