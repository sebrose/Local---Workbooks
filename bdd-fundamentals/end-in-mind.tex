\chapter*{Start with the end in mind}

\ifnotes
    This is where you introduce yourself and set expectations for the day:
    
    \begin{itemize}
        \item Introduce yourself in 30 secs: who you are, your background and how you got your experience
        \item Today (day 1) is hands-on, suitable for everybody, and won't use computers at all
        \item Tomorrow (day 2) will use computers and introduce people to \CUKE{}. People who don't code are encouraged to attend, if only for the first hour or two.
        \item Explain we use experiential techniques, because they help people learn. So, there will be no slides!
        \item Fire alarms, coffee breaks, lunch times, going home
        \item Emphasise punctuality around breaks (\& then start at the time agreed)
        \item Feedback form: \emph{https://cucumber.typeform.com/to/OAPVnn?course=<trainer>-<date>} - use your favourite link shortener.
    \end{itemize}
    
    Some trainers like to ask all attendees to introduce themselves - this can take up valuable time, so don't feel that you have to.
    
    The exercise:
    
    \begin{itemize}
        \item Ask people to write down their goal(s) on Post-It notes
        \item If they have more than one goal (which is great!) ask them to put each goal on a separate Post-It
        \item Tell them that this will help you guide them through the material over the next 2 days
        \item Ask them to stick their Post-Its onto a sheet of flip chart paper as soon as they're done
    \end{itemize}
    
    While they're doing the concept centres you will affinity map the "goals". 
    
    There will also be some more specialised goals.
    
    At the end of the day, after discussing the "Before we start" questions, but before the "Wrap up", you should go through these goals, asking the attendees if they have been sufficiently covered. If they're happy, then tick them, if not then either discuss them or park them for tomorrow. 
\fi

\ifcontent
    When we practice Behaviour-Driven Development (BDD), we take time before we start development to describe how we want the system to behave when we're done. We call this \textit{starting with the end in mind}.
    
    Think about your own requirements for this training. What outcome would you like to see on your team, or from you as a result of this training? What would success look like from your perspective?
    
    Write down at least one objective on a sticky note. You can write more if you want.
    
    Examples:
    \begin{itemize}
        \item \textit{be able to explain the benefits of BDD to one of my colleagues}
        \item \textit{know how to follow the BDD process for one story in our next sprint}
        \item \textit{decrease the number of defects we find after a story is marked as 'done'}
        \item \textit{know where I can go to learn more about this subject}
        \item \textit{learn whether my team are doing BDD right or not}
    \end{itemize}
\fi


