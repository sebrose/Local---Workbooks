\chapter*{Living documentation}

\ifnotes
    Learning outcomes:
    
    \begin{itemize}
        \item Explain how any human process is fallible. You cannot be certain that "dead" documentation is correct \& up to date
        \item Describe that 'living' documentation is self-verifying - a form of automated tests
        \item Understand that 'failing' documentation might be due to unimplemented functionality, a regression, or out of date documentation
    \end{itemize}
    
    Before moving on, this is a good time to bring these three core concepts together: 
    
    \begin{itemize}
        \item Ignorance should be dispelled as early as possible
        \item to do this we bring together representatives with different perspectives (business, dev, test) to challenge their understanding using concrete examples
        \item these examples can become tests (automated or manual) that will verify the correctness of the implementation
        \item these examples can be formulated in business readable language and become verifiably correct, living documentation
    \end{itemize}
\fi

\ifcontent
    \QandAbox{What is your project's documentation for?}{3}
    
    \QandAbox{How can you be certain that your documentation is up to date?}{3}
    
    \QandAbox{What does your team do to keep your application and its documentation in sync?}{3}
    
    \QandAbox{What do you understand by the term "Living Documentation"?}{3}
\fi