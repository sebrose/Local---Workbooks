\ifnotes

\chapter*{Concept centres}

    \section*{Learning outcomes}
    
    \begin{itemize}
        \item Describe the fundamental concepts of BDD
        \item Explain that EXAMPLES are the core concept
    \end{itemize}

    \section*{Setup}
    
    If you don't have the pre-printed posters, make sure you set aside 1h to draw the concept centres - you'll need thick markers. 
    
    Hang them around the room before training starts - using blu-tack or masking tape. Remember that groups are going to gather around them and then move around the room, so try to make sure thy are spread out and have sufficient space around them.
    
    MARK THE POSTER TUBE "DON'T THROW AWAY". The posters are valuable and are easier to carry around in the tube.
    
    To introduce the concept centres explain that:
    
    \begin{itemize}
        \item they need to self-organise into groups of 3-6 people, with a mix of roles (if possible)
        \item the groups don't need to be made up of people sitting at their table
        \item the posters introduce fundamental concepts that we'll be building on for the rest of the day
        \item feeling uncomfortable with the topics is not unexpected - it's a good way to learn
        \item they'll spend 5-6 min on each poster
        \item they'll visit each poster, so it doesn't matter where they start
        \item take their workbooks (and a pen) with them
    \end{itemize}
    
    \section*{Review}
    
    The concept centres fit together to tell a story. Review them in order with lots of input and discussion.
    
    \begin{enumerate}
        \item Discovery: examples drive discovery
        \item Examples/Rules/Tests: examples are the foundation
        \item Living documentation: examples can be formulated as documentation, which can then be automated to become self-validating
    \end{enumerate}
   
    
\fi
